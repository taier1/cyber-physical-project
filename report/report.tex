\documentclass[12pt]{article}

\usepackage{fancyhdr}
\usepackage{amssymb}
\usepackage{amsthm}
\usepackage{amsfonts}
\usepackage{mathtools}
\usepackage{array}
\usepackage{systeme}
\usepackage{geometry}
\usepackage{enumitem}
\usepackage{ dsfont }
\usepackage{listings}
\usepackage{booktabs}
\usepackage{hyperref}
\usepackage{float}
\usepackage{tikz}
\usetikzlibrary{arrows,decorations.pathmorphing,backgrounds,positioning,fit}
\usetikzlibrary{automata, positioning}
\usetikzlibrary{calc}
\usepackage[utf8]{inputenc}
  \geometry{
    left = 2.5cm,
    right = 2.5cm,
    includeheadfoot, top = 1.5cm, bottom = 1.5cm,
    headsep = 1.3cm,
    footskip = 1.2cm
  }

% Inline fraction %
\newcommand*\rfrac[2]{{}^{#1}\!/_{#2}}
\newcommand{\lincom}[4]{\mathit{{#1}{\underline{#2}}}+\mathit{{#3}{\underline{#4}}}}
\newcommand*{\QEDA}{\hfill\ensuremath{\blacksquare}}
\newcommand*{\QEDB}{\hfill\ensuremath{\square}}
\newcommand{\MU}[1]{\mathit{\underline{#1}}}

\newcommand{\DS}{\displaystyle}
\newcommand{\bb}[1]{\mathbb{#1}}

% Absolute value: \abs{x} => |x|
\newcommand{\abs}[1]{\left|#1\right|}

% Function: \func{f}{X}{Y} => f : X -> Y
\newcommand{\func}[3]{#1 : #2 \rightarrow #3}
% Function to R: \func{f}{X} => f : X -> R
\newcommand{\functoR}[2]{#1 : #2 \rightarrow \mathbb{R}}
% Function to N: \func{f}{X} => f : X -> N
\newcommand{\functoN}[2]{#1 : #2 \rightarrow \mathbb{N}}
% Function from R to R: \func{f} => f : R -> R
\newcommand{\funcR}[1]{#1 : \mathbb{R} \rightarrow \mathbb{R}}

% Limits and convergence
\newcommand{\limn}{\lim_{n \to \infty}}
\newcommand{\limh}{\lim_{h \to 0}}
\newcommand{\limsupn}{\limsup_{n \to \infty}}
\newcommand{\liminfn}{\liminf_{n \to \infty}}
\newcommand{\limx}[1]{\lim_{x \to #1}}
\newcommand{\tounif}{\xrightarrow{unif.}}

% Sums and series
\newcommand{\sumn}[1]{\sum_{k=1}^n #1}
\newcommand{\series}[1]{\sum_{k=1}^\infty #1}
\newcommand{\seriec}[1]{\sum_{k=m}^n #1}
\newcommand{\pseries}[1]{\sum_{n=0}^\infty #1}

% Intervals
% Closed-closed interval [a,b]
\newcommand{\intcc}[1]{\left[#1\right]}
% Open-closed interval (a,b]
\newcommand{\intoc}[1]{\left(#1\right]}
% Closed-open interval [a,b)
\newcommand{\intco}[1]{\left[#1\right)}
% Open-open interval (a,b)
\newcommand{\intoo}[1]{\left(#1\right)}

% Sets
\newcommand{\N}{\mathbb{N}}
\newcommand{\Z}{\mathbb{Z}}
\newcommand{\Q}{\mathbb{Q}}
\newcommand{\R}{\mathbb{R}}
\newcommand{\e}{\varepsilon}

\lhead{Alexander Fischer, Camillo Malnati, Federico Pfahler, Aldo Gabriele di Rosa}
\chead{}
\rhead{}
\rfoot{\thepage}
\cfoot{}
\lfoot{\today}
\pagestyle{fancy}

\begin{document}
  \begin{center}
    \LARGE{\textbf{MindPollution}}
  \end{center}
  \section{Introduction}
  Using bikes is an ecological way to move around cities. 
  There is a trend in cities to provide bikers preferential lanes and even bike-sharing services that aim to improve the usage of this transport over other transports methods, such as cars and motorbikes, but also public transports such as buses. 
  \emph{MindPollution} aims to collect and visualise pollution metrics inside cities by using small devices installed on all public bikes. Bikes will not only provides excellent coverage over the city due to the freedom with which the latter can move even in areas where other vehicles cannot, but it can supply data from multiple bicycles and users at the same time.

  \subsection{Initial approach}
  Our initial idea was to create a user-independent device that would have been installed on all public-bikes, and that would have required zero interactions with users. The devices would have stored the information about the pollution directly on their memory, for then transferring the data to the database service only when a user would have reached a recognised official location for bike drop-off. Due to a broken GPS sensor and no time to get a new one, we had to rethink our approach and application. In fact, to create a map of the city, we needed to have the position of the bike.
  \subsection{Selected approach}
The broken GPS sensor implied the need to find new ways to get the position of the bike in the city. The first problem is that we couldn't rely on the bike since no requirements said that the public service would have provided this information to our application. The only way to have access to the position was to develop a mobile app that provided the user's locations and transmitting the pollution information from the device directly to the phone. Being now an user-dependant device, we have decided to use the phone application to directly transfer data to the server, abandoning the idea of having data transmission stations. In the end, we also developed a web-application that visualised the collected data.
  \newpage
  \section{Implementation}

  Figure~\ref{fig:architecture} shows an overview of the architecture of our project, showing the main components.
  \begin{figure}[H]
    \centering
    \includegraphics[width=0.7\textwidth]{images/architecture.png}
    \caption{An overview of the main components in our system.}
    \label{fig:architecture}
  \end{figure}

  Shown in \autoref{fig:project-photo} a picture of the project components and how they are connected.
  \begin{figure}[H]
    \centering
    \includegraphics[width=0.7\textwidth]{images/project-all.png}
    \caption{A picture of the project components.}
    \label{fig:project-photo}
  \end{figure}
  
  \subsection{Sensors}
  To measure the air quality and pollutions levels we use two sensors that we connect directly to the Arduino board.
  \subsubsection{Air quality}
  \begin{figure}[H]
    \centering
    \includegraphics[width=0.5\textwidth]{images/mq-135.jpg}
    \caption{The MQ-135 gas sensor module}
    \label{fig:air-quality-photo}
  \end{figure}
  We measure the air quality by using a module, depicted in~\autoref{fig:air-quality-photo}, that includes a MQ-135 gas sensor (see datasheet\footnote{https://www.olimex.com/Products/Components/Sensors/Gas/SNS-MQ135/resources/SNS-MQ135.pdf}), which is sensitive to several pollutants including $NH_3$, $NO_x$, alcohol, benzene, smoke and $CO_2$.
  Its output is an aggregated level of \textit{air quality}, either as an analog voltage or as a digital output based on a threshold. An higher value implies that the density of the detected gases is higher, and therefore the air quality is worse.

  This module requires a supply voltage of $5\ volts$, and we connect its analog output to an analog input pin (i.e. $A0$) of our Arduino Nano board.

  \subsubsection{Particulate matter (PM2.5/PM10)}
  \begin{figure}[H]
    \centering
    \includegraphics[width=0.5\textwidth]{images/pm-sensor.jpg}
    \caption{The Nova Fitness SDS011 PM2.5/PM10 sensor}
    \label{fig:pm-sensor-photo}
  \end{figure}
  Another sensor, shown in~\autoref{fig:pm-sensor-photo}, that we use to measure the pollution is the Nova Fitness SDS011 (see datasheet\footnote{https://cdn-reichelt.de/documents/datenblatt/X200/SDS011-DATASHEET.pdf}), which measures the density of particulate matter (i.e. PM2.5 and PM10) in the air. Its range of measured values is between $0$ and $999\mu g/m^3$.

  To output those values, the module has a serial interface based on the UART communication protocol, described in \autoref{tab:uart-protocol}, and the rate of transmission of the measurements is $1Hz$. We connected the output pin of this serial interface to the serial input pin of our Arduino Nano board. Furthermore, similarly to the air quality sensor, the module requires a supply voltage of $5\ volts$.

  \begin{table}[H]
    \centering
    \begin{tabular}{r | l | l}
      \# & Name & Content\\\toprule
      0 & Header & AA\\
      1 & Commander No. & C0\\
      2 & DATA 1 & PM2.5 Low Byte\\
      3 & DATA 2 & PM2.5 High Byte\\
      4 & DATA 3 & PM10 Low Byte\\
      5 & DATA 4 & PM10 High Byte\\
      6 & DATA 5 & ID Byte 1\\
      7 & DATA 6 & ID Byte 2\\
      8 & Checksum & Checksum\\
      9 & Tail & AB\\
    \end{tabular}
    \caption{The bytes sent over the serial interface}
    \label{tab:uart-protocol}
  \end{table}

  In order to extract the correct values of bot PM2.5 and PM10 the following formula should be applied to the content of \autoref{tab:uart-protocol}:

  \begin{itemize}
    \item PM2.5 value: PM2.5 ($\mu g/m^3$)  =  ((PM2.5  High  byte  *256)  +  PM2.5 low byte)/10
    \item PM10 value: PM10 ($\mu g/m^3$)  =  ((PM10  high  byte*256)  +  PM10  low byte)/10
    \item Checksum: Checksum=DATA1+DATA2+...+DATA6
  \end{itemize}

  \subsection{Arduino code}

  \subsection{Android application}
  The goal of the Android application is to act as a \textit{gateway} between the Arduino board that collects pollution data from sensors and the web application that stores and visualizes the aggregated data. While forwarding pollution measurements, it also adds the current location coordinates obtained from the phone's GPS sensor.

  The application runs on a mobile phone (in our demo a Huawei P10 lite running Android 8.0) that is connected via Bluetooth to the Arduino Nano 33 BLE board, as well as with an active GSM connection.
  Its source code is based on an example application, \verb|BluetoothLeGatt|\footnote{Also on Github: https://github.com/android/connectivity-samples/tree/master/BluetoothLeGatt}, which is bundled with the official Android SDK, that allows to scan for Bluetooth Low Energy devices and connect to them. 

  What we added to that code was to, on every message received from the Arduino board, retrieve the location coordinates (code is adapted from a blog post\footnote{https://medium.com/@ssaurel/getting-gps-location-on-android-with-fused-location-provider-api-1001eb549089}) and send that data together with the pollution information to a web server as a JSON object by performing a POST request. An example of such JSON object is shown in Figure~\ref{lst:json-packet}. On the other hand, the message sent from the board is structured as follows: \verb|123,12,5;|. It contains the measurements from the air quality sensor, as well as the two measured values from the PM2.5/10 sensor.

  \begin{figure}[H]
    \centering
    \includegraphics[width=0.3\textwidth]{images/android1.jpg}
    \includegraphics[width=0.3\textwidth]{images/android2.jpg}
    \caption{The two main views of the application: device list (left), device information (right)}
    \label{fig:app-scan-list}
  \end{figure}

  %TODO: add screenshots of app

  \begin{figure}[h]
    \centering
    \begin{verbatim}
{
  "createdAt": "2019-12-13T09:22:49.824Z",
  "lat": 46.0037,
  "long": 8.9511 ,
  "bikeId" : "00002add-0000-1000-8000-00805f9b34fb",
  "pm10": 5,
  "pm25": 12,
  "airQuality": 123
}
    \end{verbatim}
  \caption{An example JSON object containing the pollution measurements and location}\label{lst:json-packet}
  \end{figure}

  \subsection{Web application}
  \newpage
  \section{Conclusions}
  

\end{document}
